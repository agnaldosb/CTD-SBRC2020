{\color{red}--------------------------------- REMOVER AO FINAL -------------------------------}

Orientações do CTD do SBRC 2020

O texto deve incluir, obrigatoriamente, o \textcolor{green}{título da tese ou dissertação, nomes do autor, orientador e coorientador(es) com as respectivas afiliações,} \textcolor{red}{resumo, abstract, caracterização do problema e motivação, objetivos} e \textcolor{green}{contribuições, trabalhos relacionados, resultados obtidos e subprodutos do trabalho, tais como artigos científicos, patentes, programas registrados e quaisquer outros elementos que permitam qualificar o trabalho.}

Cada submissão deverá conter três arquivos: \textcolor{red}{(1) resumo estendido de até oito páginas do trabalho;} \textcolor{green}{(2) ata da defesa do trabalho; e (3) texto completo da tese ou dissertação. A ata da defesa deve conter o local e a data da defesa, o nome do orientador, membros da banca e um parecer que indique explicitamente que o candidato defendeu e teve o trabalho aprovado.}

{\color{red}------------------------------------------------------------------------------------------------}

\section*{Ideias}

Logo, esses dados devem ser preservados do acesso não autorizado.
%~\cite{cole2017risk}. 


Esses serviços geralmente coletam e entregam dados por contatos oportunísticos, enquanto as interações permitem às pessoas próximas geograficamente comunicarem entre si~\cite{garyfalos2008coupons}.

As contribuições desta dissertação de mestrado são: a análise do uso das técnicas de confiança em redes não estruturadas (IoT, MANETs e P2P); a proposta e a especificação do sistema STEALTH; e a a avaliação e análise da eficácia do STEALTH na disseminação de dados sensíveis de maneira robusta diante de situações emergenciais realísticas.

Porém, a entrega de dados implica seu compartilhamento, demandando observar a frequência, o local e o conteúdo a ser disseminado~\cite{sudhindra2014}. Em razão das suas caraterísticas e para garantir o seu funcionamento, muitos destes serviços naturalmente exigem a criação e manutenção de redes locais ou globais estabelecidas dinamicamente.
 
A robustez e a garantia da entrega dos dados estão associadas com relação ao espaço e tempo, bem como questões sociais, que não têm sido muito consideradas. O aumento expressivo do volume de dados gerados pelas pessoas passou a incorporar as informações temporais relativas a esses dados, o que permitiu o surgimento das redes dinâmicas. Essas redes levam em conta que as interações sociais das pessoas evoluem ao longo do tempo, assim como sua mobilidade, o que faz que sua topologia se modifique com o passar do tempo~\cite{rossetti2017community}. O acesso ubí\-quo aos dados em redes dinâmicas pode ser impactado por diversos fatores, tais como o momento em que ele ocorre, as interações entre as pessoas nesse instante e a localização.

A interação entre dispositivos computacionais móveis e as pessoas tem sido intensificada e estabelecido redes locais temporárias, onde se trocam informações com diferentes propósitos e normalmente por um certo período de tempo. Essas redes podem ser estabelecidas através de redes ad hoc sem fio. Embora as redes de telefonia móvel, por exemplo, ofereçam uma cobertura cada vez maior nas cidades, elas não permitem uma comunicação direta entre dispositivos, podendo influenciar no tempo de atendimento de eventos críticos. Dispositivos móveis, como \textit{smartphones}, coletam vários tipos de dados a fim de apoiar melhorias nos serviços de vigilância, transportes e saúde, entre outros. Particularmente nos serviços de saúde, os \textit{smarphones} possibilitam a conexão dos dispositivos médicos das pessoas à Internet~\cite{williams2016perfect}, podendo comprometer sua segurança. Os serviços de saúde em redes (\textit{e-health})  auxiliam  a acompanhar remotamente o estado de saúde dos cidadãos, além de contribuir para mudar comportamentos nocivos ao seu bem estar. Em uma área urbana, por exemplo, onde pessoas deslocam-se a pé pelas ruas, uma delas pode sentir-se mal, vindo a ser auxiliada por aquelas ao seu redor. Nesse contexto, alertas médicos devem ser transmitidos imediatamente~\cite{movassaghi2014wireless} e com uma latência máxima de 125ms~\cite{ieee2012}, pois perdas ou atrasos acarretam consequências graves à saúde dos pacientes~\cite{latre2011survey}.

Atualmente, há  diversas plataformas para interação social das pessoas, onde aspectos das suas relações são observados e empregados no controle das trocas de dados. Muitas dessas plataformas online potencializam a distribuição de informações, causam vazamentos de dados e comprometem sua segurança, como aconteceu com o Facebook ~\cite{FacebookLeakage}. Logo, um serviço seguro (\textit{safety}) de disseminação de dados em redes entrega os dados às pessoas corretas e evita vazamentos~\cite{lima2009survey}. Embora diversos trabalhos na literatura tratem essa segurança em redes não estruturadas nos contextos de IoT~\cite{al2017trust,bao2013scalable}, MANETs~\cite{mannes2012quorum} e P2P~\cite{vasilomanolakis2017trust},  as soluções geralmente se destinam a ambientes centralizados \rev{e necessitam conhecer as} interações anteriores para a tomada de decisões no manuseio dos dados. Poucas pesquisas voltam-se aos ambientes onde interações anteriores são desconhecidas (\textit{Zero-Knowledge}~\cite{feige1988zero}), nos quais há informações apenas de interações atuais. Dessa forma, essas soluções não são adequadas aos ambientes urbanos  \rev{dinâmicos} e esparsos, pois consideram \rev{a existência prévia de} uma infraestrutura de rede %sempre disponível para atender à disseminação de dados. 

O controle da disseminação dos dados em redes pretende garantir a entrega de dados às entidades corretas e no tempo adequado. Alguns trabalhos encontrados na literatura empregam confiança como critério para esse controle~\cite{al2017trust}. \rev{Entende-se por confiança como a vontade de uma pessoa de arriscar-se, baseada numa crença subjetiva de que aquele em quem ela confia exibirá um comportamento confiável. A confiança social provém das relações entre essas pessoas no ambiente das redes sociais.~\cite{cho2015survey}} e é empregada para agrupar nós da rede, estabelecendo comunidades de interesse baseadas em aspectos sociais dos proprietários dos dispositivos de rede e de suas relações~\cite{bao2013scalable}. Embora adequados às redes não estruturadas, esses trabalhos empregam reputação e recomendação para avaliar a confiança, que são técnicas dependentes de informações relativas às interações passadas entre os dispositivos. 

%\noindent
\begin{minipage}{.3\linewidth}
\centering
\begin{equation}
T_{xy}^{I} = \frac {|I_x \cap I_y|}{|I_x|}
\label{eq:communityTrust}
\end{equation}
\end{minipage}
% Valores possíveis 
\begin{comment}
\quad
\begin{minipage}{.7\linewidth}
\begin{equation}
\footnotesize
T_{xy}^{I} = \left \{
\begin{array}{cl} 
\rev{0}, & se  \; \; I_y \not\supset \{sa\Acute{u}de\} \\
]0,1[,	& se \; \; I_x \cap I_y \neq 0 \; \; e \; \; I_x \neq I_y \; e \; \{sa\Acute{u}de\} \subset I_x \cap I_y\\
1, 	& se \; \; I_x = I_y \; \; e \; \; \{sa\Acute{u}de\} \subset I_x \cap I_y
\end{array}
\right.
\label{eq:valuesTCoI}
\end{equation}
\end{minipage}
\end{comment}
%próxima equação
\begin{minipage}{.3\linewidth}
\centering
\begin{equation}
T_{xy}^{Skill} = Sim_y
\label{eq:SkillTrust}
\end{equation}
\end{minipage}
\begin{comment}
\vspace{0.4cm}
\quad
\begin{minipage}{.7\linewidth}
\begin{equation}
T_{xy}^{Skill} = \left \{
\begin{array}{cl} 
0,	& se \; s_y \; \in \{outras\} \\
]0,1[,	& se \; s_y \; \notin \{outras, m\Acute{e}dico\} \\
1, 	& se \; s_y \; \in \{m\Acute{e}dico\}
\end{array}
\right.
\label{eq:valuesTSkill}
\end{equation}
\end{minipage}
\end{comment}
\begin{minipage}{.3\linewidth}
\centering
\begin{equation}
T_{xy} = \frac{T_{xy}^{I} + T_{xy}^{Skill}}{2}
\label{eq:totalTrust}
\end{equation}
\end{minipage}